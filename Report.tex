\documentclass[11pt,a4paper]{extarticle}
\usepackage{geometry}
 \geometry{
 a4paper,
 total={160mm,237mm},
 left=25mm,
 top=25mm,
 }
\usepackage[utf8x]{inputenc}
\usepackage[english]{babel}
\usepackage{bbm}
\usepackage[usenames]{color}
\usepackage{hyperref}
\usepackage{colortbl}
\usepackage{amsmath}
\usepackage{amssymb}
\usepackage[pdftex]{graphicx}
\usepackage{amsthm}
\usepackage{caption}
\usepackage[ruled,vlined]{algorithm2e}
\usepackage{authblk}

 
\title{Course project \\
"Optimization approaches to community detection"}
\author{ Marina Danilova, Alexander Podkopaev, Nikita Puchkin, Igor Silin }

\begin{document}

\maketitle

\section{Introduction to community detection}
Wide range of real-life objects can be represented in terms of graphs. When working with graphs, it may be very useful to find groups of nodes, such that there are much more edges within these groups, than between groups. Such groups are called clusters or communities, and the problem of finding these groups is called community detection. A lot of algorithms for community detection were developed and huge part of them uses optimization method. In this work we want to present a number of methods, that allows to reduce the community detection problem to an optimization problem.

Let's introduce some notations.
We consider undirected unweighted graphs without loops with $n$ nodes and $m$ edges.
The nodes are enumerated as $\{ 1, ..., n\}$. 
Graph is given by its $n \times n $ adjacency matrix $A$.
Degree of the node $i$ is $d_i$. 

The number of clusters is $k$ (some algorithms require to specify this number).
The clusters are denoted as $\{ \mathcal{C}_1, ..., \mathcal{C}_k\}$.
Cluster sizes are $ |\mathcal{C}_1|, ..., |\mathcal{C}_k|$.
Sometimes it's convenient to describe cluster structure in terms of labeling $z$: $z(i)$ is the cluster containing node $i$, i.e. $i \in \mathcal{C}_{z(i)}$.
				
\section{Stochastic block model}
	In this section we discuss the stochastic block model, which is common widely used tool for generating graphs with special structure.
	Some of the algorithms that we used (natural conjugate gradients method, semidefinite relaxation method) works exactly with this model.
		
		Again we consider a graph with $n$ nodes.
		Suppose we want to create some graph with $k$ clusters. 
		Let the structure be given by labeling $z: \{ 1,...,n\} \rightarrow \{ 1,...,k\}$.
		Also we introduce a symmetric matrix of inter-cluster probabilities $P = ||P_{ij}||_{i,j = 1}^k \in [0; 1]^{k\times k}$. Elements of this matrix give probability with which an edge between vertex of one cluster and vertex of another or the same cluster is generated.
		
		Using labeling $z$ and inter-cluster probabilities $P$ we can construct SBM parameter matrix $\theta$:
		
		\begin{equation}
		\begin{aligned}
		\theta_{ij} = P_{z(i)\;z(j)}\;\;\forall i \neq j, \;\;\;\;\theta_{ii} = 0\;\;\forall i.
		\nonumber
		\end{aligned}
		\end{equation}
		Element $\theta_{ij}$ means the probability of generating an edge between $i$-th and $j$-th nodes.
		So a graph, or more specifically its adjacency matrix $A$, is generated in the following way:
		\begin{equation} \label{eq}
		\begin{aligned}
		a_{ij} \sim Bernoulli(\theta_{ij}) ,\; a_{ji} = a_{ij} \;\; \forall i > j,\;\;\;\;\;a_{ii} = 0\;\; \forall i,
		\end{aligned}
		\end{equation}
		where all $A_{ij}$ are independent.
		In this work we consider unweighted undirected graph without self-loops.

		The goal of community detection is to reconstruct labeling $z$, while also one can be interested in estimation of parameter matrix $\theta = [\theta_{ij}]_{i,j=1}^n$. In order to find this structure, we use only given graph.
		
		As an example, take a look on the simplest case with two clusters ($k = 2$). Let these clusters be of equal size $n/2$. Also suppose that an edge between two nodes of the same cluster appears with probability $a$ while an edge between two nodes of different clusters appears with probability $b$, where $a > b$.
		Described case corresponds to the following labeling $z$ and inter-cluster probabilities $P$:
		\begin{equation}
		\begin{aligned}
		z = [\underbrace{1, ..., 1}_{n/2}, \underbrace{2, ..., 2}_{n/2}]^T, \;\;\; P = \begin{bmatrix} a & b \\ b & a \end{bmatrix}.
		\nonumber
		\end{aligned}
		\end{equation}
		One can easily construct SBM parameter matrix $\theta$ based on $z$ and $P$. Now the graph generating process is clear.

\section{Algorithms}
	
	\subsection{Semidefinite relaxations}
		%\numberwithin{equation}{subsection}

In this method it's very convenient to introduce clustering matrix $X$ of size $n\times n$ with $x_{ij} = \mathbbm{1}\{ z(i) = z(j) \}$.
We denote space of all matices with such structure as $\mathcal{X}$.
                
Next we want to formulate an optimization problem.
We consider likelihood for special case of stochstic block model:
\begin{equation}
    \begin{aligned}
        \mathcal{L}(X) = trace(AX).
    \nonumber
    \end{aligned}
\end{equation}
Then we write maximum likelihood method:
\begin{equation}
    \begin{aligned}
    &maximize\;\;\mathcal{L}(X)\\
    &s.t. \;\;X \in \mathcal{X}
    \nonumber
    \end{aligned}
\end{equation}
This problem is NP-hard combinatorial optimization problem. The idea is to relax constraints.
So, we just relax $X \in \mathcal{X}$ and get semidefinite program:
\begin{equation}
    \begin{aligned}
    &maximize\;\;trace(AX)\\
    &s.t. \;\;X \text{ is positive semi-definite}, \\
    &\;\;\;\;\;\;\; X \geq 0, \\
    &\;\;\;\;\;\;\; diagonal(X) = e, \\
    &\;\;\;\;\;\;\; Xe = \frac{n}{k}e,
    \nonumber
    \end{aligned}
\end{equation}
where $e = (1,\ldots,1)^T$.
And finally we solve this problem with SDP solvers implemented in cvx. We call this formulation ''SDP1''.

In the experiments we will consider one more possible semidefinite relaxation, but not so tight one. We call it ''SDP2''.

\begin{equation}
    \begin{aligned}
    \nonumber
    \end{aligned}
\end{equation}

	\subsection{Natural conjugate gradients method}
		
It is assumed, that each component of labelling $z$ has a polynomial distribution with parameter $\pi$
and each element of $A$ has a Bernoulli distribution with parameter $P_{z(i)z(j)}$:

\begin{equation}
    \begin{aligned}
    \nonumber
        & z_i \sim \text{Poly}(\pi), \quad \pi^T = (\pi_{1}, \dots, \pi_{k}), \\
        & a_{ij} \sim \text{Bernoulli}(P_{z(i)z(j)}), \quad i,j = \overline{1, n}, \\
    \end{aligned}
\end{equation}
A Bayesian approach is used to estimate the most probable configuration of $z$ given an adjacency matrix $A$

\begin{equation}
    \begin{aligned}
    \label{maxz}
    z^* = \arg\max\limits_z p(z | A) = \arg\max\limits_z \iint p(z, \pi, P | A) d\pi dP
    \end{aligned}
\end{equation}
It treats parameters $\pi$ and $\theta_{ij}$ as random variables with following prior distributions

\begin{equation}
    \begin{aligned}
    \nonumber
        & \pi \sim \text{Dirichlet}(\alpha) \\
        & P_{ii} \sim \text{Beta}(\beta), \quad i = \overline{1,k}, \\
    \end{aligned}
\end{equation}
where $\alpha$ and $\beta$ are predefined hyperparameters
and $P_{ij}$, $i\neq j$ are set to equal to a small constant $\varepsilon$.
Futhermore, to tackle the intractable integration in \ref{maxz} a restriction on the family of factorized distributions is considered

\begin{equation}
    \begin{aligned}
    \nonumber
    \mathcal Q = \{q:\, q(z, \pi, P) = q(\pi)q(P) \prod\limits_i q(z_i) \},
    \end{aligned}
\end{equation}
where
\begin{equation}
    \begin{aligned}
    \nonumber
    & q(z_i): z_i \sim \text{Poly}(\tilde \pi) \\
    & q(\pi): \pi \sim \text{Dirichlet}(\tilde \alpha) \\
    & q(P): P_{ii} \sim \text{Beta}(\tilde\beta_i)
    \end{aligned}
\end{equation}

The optimal distribution $q^* \in \mathcal Q$, that approximates the true posterior $p(z, \pi, P | A)$ minimizes the Kullback-Leibler divergence

\begin{equation}
    \begin{aligned}
    \label{minq}
    q^* = \arg\min\limits_{q \in \mathcal Q} \text{KL} \left( q \| p(z, \pi, P | X) \right)
    \end{aligned}
\end{equation}

Define
\begin{equation}
    \begin{aligned}
    \mathcal L(q) = \sum\limits_z \iint q(z, \pi, P) \log \frac{p(A, z, \pi, P)}{q(z, \pi, P)} d\pi dP
    \end{aligned}
\end{equation}
Since

\begin{equation}
    \begin{aligned}
    \nonumber
    \mathcal L(q) + \text{KL} \left( q \| p(z, \pi, P | X) \right) = \log p(A)
    \end{aligned}
\end{equation}
the minimization problem \ref{minq} can be equivalently solved by maximizing $\mathcal L(q)$

\begin{equation}
    \begin{aligned}
    \label{maxq}
    q^* = \arg\max\limits_{q \in \mathcal Q} \mathcal L(q)
    \end{aligned}
\end{equation}

Denote $\tilde\Pi = \| \tilde\pi_{ij} \|$, $i = \overline{1, n}$, $j = \overline{1, k}$.
Given $\tilde\Pi$ values of parameters, that maximize $\mathcal L(q)$ can be found according to formulas

\begin{equation}
    \begin{aligned}
    \label{alpha_beta}
    & \tilde\alpha = \alpha + \tilde\Pi^T 1 \\
    & \tilde \beta_i = \beta + \frac12 \left( \tilde\Pi_{\cdot i}^T A \tilde\Pi_{\cdot i}, \tilde\Pi_{\cdot i}^T \overline A \tilde\Pi_{\cdot i} \right)^T, \quad i = \overline{1, k},
    \end{aligned}
\end{equation}
where $1$ denotes an all-ones vector, $\overline A = 11^T - I - A$, and $\tilde\Pi_{\cdot i}$ stands for the $i$\/-th column of the matrix $\tilde\Pi$.
A corresponding value of log-likelihood is equal to
\begin{equation}
    \begin{aligned}
    \label{L}
    \mathcal L(\tilde\Pi) = \sum\limits_{i < j}  \tilde\Pi_{\cdot i}^T A \tilde\Pi_{\cdot j} \log\varepsilon + \tilde\Pi_{\cdot i}^T \overline A \tilde\Pi_{\cdot j}\log(1 - \varepsilon) - \sum\limits_{i,j} \tilde\pi_{ij} \log\tilde\pi_{ij} + \log\frac{\mathcal B(\tilde\alpha)}{\mathcal B(\alpha)} + \sum\limits_i \log\frac{\mathcal B(\tilde\beta)}{\mathcal B(\beta)},
    \end{aligned}
\end{equation}
where $\tilde\alpha = \tilde\alpha(\tilde\Pi)$ and $\tilde\beta = \tilde\beta(\tilde\Pi)$ can be found from \ref{alpha_beta}, $\mathcal B(\alpha) \triangleq \frac{\Gamma\left( \sum\limits_i \alpha_i \right)}{\prod\limits_i \Gamma(\alpha_i)}$ and $\Gamma$ is gamma-function.
Now the maximization problem \ref{maxq} can be reformulated as follows
\begin{equation}
    \begin{aligned}
    \begin{cases}
        \mathcal L(\tilde\Pi) \longrightarrow \max \\
        \sum\limits_{j} \tilde\pi_{ij} = 1, \quad i = \overline{1,n}
    \end{cases}
    \end{aligned}
\end{equation}
One can use reparametrization

\begin{equation}
    \begin{aligned}
    \label{pi}
    & \tilde\pi_{ij} = e^{\theta_{ij} - \mathcal A_i}, \quad i = \overline{1, n}, j = \overline{1, k-1}, \\
    & \tilde\pi_{ik} = e^{-\mathcal A_i}, \quad i = \overline{1, n}, \\
    & \mathcal A_i = \log \left( 1 + \sum\limits_{j=1}^{k-1} e^{\theta_{ij}} \right), \quad \quad i = \overline{1, n} \\
    \end{aligned}
\end{equation}
and obtain a problem of unconstrained maximization

\begin{equation}
    \begin{aligned}
    \label{max_theta}
    \mathcal L(\theta) \longrightarrow \max
    \end{aligned}
\end{equation}

The problem \ref{max_theta} can be solved via natural conjugate gradient method.
Namely, given an initial value $\theta^{(0)}$, one iteratively finds optimal value of $\theta$ as follows
\begin{equation}
    \begin{aligned}
    \nonumber
    \theta^{(t+1)} = \theta^{(t)} + \lambda^{(t)} d^{(t)}, \quad t = 0, 1, 2, \dots
    \end{aligned}
\end{equation}
Here $d^{(t)}$ is so called natural conjugate gradient.
$d^{(t)}$ can be found according to formulas

\begin{equation}
    \begin{aligned}
    \label{d}
    d^{(t)} =
    \begin{cases}
        g^{(t)}, \quad t = 0 \\
        g^{(t)} + \frac{\| g^{(t)} \|_\theta^2}{\| g^{(t-1)} \|_\theta^2} d^{(t-1)}, \quad t > 0,
    \end{cases}
    \end{aligned}
\end{equation}
where $g^{(t)}$ is a natural gradient of $\mathcal L(\theta)$.
$\| \cdot \|_\theta$ stands for the norm with respect to Riemannian metrics
\begin{equation}
    \begin{aligned}
    \nonumber
    G(\theta) = \text{diag} \left( \mathcal I(\theta_1), \dots, \mathcal I(\theta_N) \right),
    \end{aligned}
\end{equation}
where $\theta_i = (\theta_{i1}, \dots, \theta_{i,k-1})^T$, $i = \overline{1, n}$ and $\mathcal I(\theta_i)$ is a Fischer information.
This method is nothing else but a conjugate gradient method in a Riemannian space with metrics $G(\theta)$.
Given $\theta$, $g$ and $\|g\|_\theta$ can be found as follows
\begin{equation}
    \begin{aligned}
    \label{ng}
    & g = \nabla_{\tilde\Pi} \mathcal L(\tilde\Pi) =
    \begin{pmatrix}
        (I, -1) \nabla_{\tilde\pi_1} \mathcal L(\tilde\Pi) \\
        \vdots \\
        (I, -1) \nabla_{\tilde\pi_n} \mathcal L(\tilde\Pi)
    \end{pmatrix} \\
    & \| g \|_\theta^2 = \sum\limits_{i = 1}^n \left( \nabla_{\tilde\pi_1} \mathcal L(\tilde\Pi) \right)^T \left( \text{diag}(\tilde\pi_i) - \tilde\pi_i^T \tilde\pi_i \right) \left( \nabla_{\tilde\pi_1} \mathcal L(\tilde\Pi) \right),
    \end{aligned}
\end{equation}
where $\tilde\Pi = \tilde\Pi(\theta)$
and
\begin{multline}
    \label{L_prime}
    \frac{\partial \mathcal L(\tilde\Pi)}{\partial \pi_{ij}} = \sum\limits_{l \neq j} \left( A_{i\cdot}\tilde\Pi_{\cdot l} \log\varepsilon + \overline A_{i\cdot}\tilde\Pi_{\cdot j} \log(1 - \varepsilon) \right) + \\
    \left( A_{i\cdot}\tilde\Pi_{\cdot j}, \overline A_{i\cdot}\tilde\Pi_{\cdot j} \right) \nabla\log\mathcal B (\tilde\beta_{j}) + \psi(\tilde\alpha_j) - \log\tilde\pi_{ij} - 1,
\end{multline}
where $\psi(\cdot)$ is digamma function.

The final algorithm is given in \ref{ncg_alg}.

\begin{algorithm}[H]
    \caption{Natural Conjugate Gradient}
    \label{ncg_alg}
	\SetAlgoLined

	\KwIn{adjacency matrix $A$, maximum number of clusters $k$, tolerance $\eta$, maximum number of iterations $t_{\max}$}
	\KwOut{array of predicted labels Z}
	Initialize $\theta$; \\
    $\mathcal L_{old} \leftarrow -\infty$; \\
    $\lambda \leftarrow 1$; \\

    \For{$t$ in range ($t_{\max}$)}
    {
        Calculate $\tilde\Pi$ using \ref{pi}; \\
        Calculate $\tilde\alpha$, $\tilde\beta$ using \ref{alpha_beta}; \\
        Calculate $\mathcal L$ using \ref{L}; \\

        \uIf{
                $0 \leq \frac{\mathcal L - \mathcal L_{old}}{|\mathcal L|} < \eta$
            }
            {
                \bf break
            }
        \eIf{
                $\eta \geq 0$
            }
            {
                Update $d$ using \ref{d}, \ref{ng}, \ref{L_prime}; \\
                $\theta_{old} \leftarrow \theta$ \\
                $\theta \leftarrow \theta_{old} + \lambda d$; \\
                $\mathcal L_{old} \leftarrow \mathcal L$; \\
            }
            {
                $\lambda \leftarrow \frac\lambda2$ \\
                $\theta \leftarrow \theta_{old} + \lambda |\eta| d$ \\
            }
    }

    $Z = \left( \arg\max\limits_{1\leq j \leq k} \tilde\pi_{1j}, \dots, \arg\max\limits_{1\leq j \leq k} \tilde\pi_{nj} \right)$

    \Return $Z$	
\end{algorithm}

Initial value of $\theta$ was generated from a standard normal distribution $\mathcal N(0, 1)$.
Probability of occurrence an inter-cluster edge $\varepsilon$ was set to $10^{-10}$.
The maximal number of iterations and relative tolerance were taken equal to $100$ and $10^{-6}$ respectively.
Examples of performance of the natural conjugate gradient method can be found, for example, in \cite{ncg}.
	\subsection{Modularity-based methods}
		%\numberwithin{equation}{subsection}
In this subsection we discuss the method described in \cite{modularity}.
The method is based on the concept of modularity. To introduce it, we first of all compute the fraction of edges which lie within communities:
    \begin{equation}
        \begin{aligned}
            \frac{1}{2m} \sum\limits_{i,j=1}^{n} a_{ij} \cdot \mathbbm{1}\{ z(i) = z(j) \}.
        \nonumber
        \end{aligned}
    \end{equation}
We are also interested in the expected number with respect to the configuration model. The configuration model is a randomized realization of a particular network. Given a network with $n$ nodes, where each node $i$ has a node degree $d_i$, the configuration model cuts each edge into two halves, and then each half edge, called a stub, is rewired randomly with any other stub in the network even allowing self loops. Thus, even though the node degree distribution of the graph remains intact, the configuration model results in a completely random network and the expected fraction of edges which lie within communities:
    \begin{equation}
        \begin{aligned}
            \sum\limits_{i,j=1}^{n}\frac{d_i}{2m}\frac{d_j}{2m} \cdot \mathbbm{1}\{ z(i) = z(j) \}.
            \nonumber
        \end{aligned}
    \end{equation}
Finally, \textbf{Modularity} is the difference between two previous fractions:
    \begin{equation}
        \begin{aligned}
            Q(z) = 
            \frac{1}{2m} \sum\limits_{i,j=1}^{n} \left(a_{ij} - \frac{d_i\cdot d_j}{2m}\right) \cdot \mathbbm{1}\{ z(i) = z(j) \}.
        \nonumber
        \end{aligned}
    \end{equation}
Modularity can take values in interval $[-1/2;\;1)$.
When modularity is large, it means that the network is far from its average state and there the density of edges within communities is higher.
But finding exact maximum of modularity is NP-hard problem because we need to search through exponential number of clusterings.
So, the goal of modularity-based algorithms is to approximately maximize modularity over all possible partitions of the graph.

Different ideas for maximization can be used, but we focus on simple greedy algorithm. To apply this method, it will be more convinient for us to rewrite the definition of modularity in the following form:
 \begin{equation}
    \begin{aligned}
    Q = \sum\limits_{q=1}^{k} e_{{\mathcal{C}_q}{\mathcal{C}_q}} - b^2_{\mathcal{C}_q}, 
    \nonumber
    \end{aligned}
\end{equation}
where 
\[
    e_{{\mathcal{C}_q}{\mathcal{C}_p}} = \frac{1}{2m} \sum\limits_{i,j=1}^{n} a_{ij} \cdot \mathbbm{1}\{ i \in {\mathcal{C}_q}, j \in {\mathcal{C}_p}\}
\]
and 
\[
b_{\mathcal{C}_q} = \sum\limits_{p=1}^{k} e_{{\mathcal{C}_q}{\mathcal{C}_p}}.
\]
One can easily check that this formulation is equivalent to the previous one.

Now let's describe the idea of the greedy maximization of the modularity for given graph.
Initially we put each node in its own cluster, i.e. $\forall i \rightarrow \mathcal{C}^{(0)}_i = \{ i \}$.
Then we start our iterations. On $t$-th iteration we look at current clusters $\mathcal{C}^{(t-1)}_q$ and look for a pair of clusters, union of which gives us the maximal gain $\Delta Q$ of modularity. 
Namely, if we join clusters ${\mathcal{C}^{(t-1)}_q}$ and ${\mathcal{C}^{(t-1)}_p}$ together and form new cluster $\mathcal{C}^{(t)}_{(qp)}$ and leave all other clusters as they are, i.e. $\mathcal{C}^{(t)}_s = \mathcal{C}^{(t-1)}_s \; \forall s\neq q,p$, then $e$ and $b$ can be recalculated as follows:
\begin{equation}
    \begin{aligned}
        & e_{\mathcal{C}^{(t)}_{(qp)} \mathcal{C}^{(t)}_s} = e_{{\mathcal{C}^{(t-1)}_q}{\mathcal{C}^{(t-1)}_s}} + e_{{\mathcal{C}^{(t-1)}_p}{\mathcal{C}^{(t-1)}_s}} \;\; \forall s \neq q,p,\\
        & e_{\mathcal{C}^{(t)}_{(qp)} \mathcal{C}^{(t)}_{(qp)}} = e_{{\mathcal{C}^{(t-1)}_q}{\mathcal{C}^{(t-1)}_q}} +
        e_{{\mathcal{C}^{(t-1)}_p}{\mathcal{C}^{(t-1)}_p}} + e_{{\mathcal{C}^{(t-1)}_q}{\mathcal{C}^{(t-1)}_p}} + e_{{\mathcal{C}^{(t-1)}_p}{\mathcal{C}^{(t-1)}_q}},\\
        & b_{\mathcal{C}^{(t)}_{(qp)}} = b_{\mathcal{C}^{(t-1)}_{q}} + b_{\mathcal{C}^{(t-1)}_{p}}.
    \nonumber
    \end{aligned}
\end{equation}
All other elements don't change. Hence, according to our formula for modularity we have:
\begin{equation}
    \begin{aligned}
        &Q^{(t)} = Q^{(t-1)} + \Delta Q,\\
        \Delta Q = 
        &e_{{\mathcal{C}^{(t-1)}_q}{\mathcal{C}^{(t-1)}_p}} + e_{{\mathcal{C}^{(t-1)}_p}{\mathcal{C}^{(t-1)}_q}} 
        - 2b_{\mathcal{C}^{(t-1)}_{q}}  b_{\mathcal{C}^{(t-1)}_{p}}.
    \nonumber
    \end{aligned}
\end{equation}
If we keep and recalculate all $e$ and $b$ during the iterations, we can determine pair of current clusters that gives maximal $\Delta Q$ effectively, since for any possible pair calculating $\Delta Q$ takes constant time (actually, more advanced approaches for more effective search of such pair were developed, e.g. one can use heap structure). So, we choose two clusters that we want to unite and unite them. We also have the value of modularity for current clustering. We also need to recalculate $e$ and $b$ as shown above.

One can consider this sequential uniting of clusters as the process of building a dendrogarm, which is a tree, that shows in which order cluster were joined.

After we did $(n-1)$ iterations, we have one cluster that contains all nodes. So as a final clustering we need to choose clustering from iteration $t$ with the maximal modularity $Q^{(t)}$.

For technical details of implementation see code in ''Lib/modularity.py''.

Advantages of this procedure are that it gives some partition in any case and doesn't require to specify the number of clusters that we are looking for. It's good, because in real-world problems we rarely know the exact number of communities that we want to detect. At the same time, there are no theoretical results that guarantee some good performance of the method.



\begin{equation}
    \begin{aligned}
    \nonumber
    \end{aligned}
\end{equation}

	\subsection{Spectral method}
		
\numberwithin{equation}{subsection}

 Here we discuss methods proposed in \cite{spectral}. The intuition behind of clustering is to separate points in different groups according to their similarities. The initial problem can be restated as a graph partitioning problem: one wants to find a partition of the graph such that there are not many edges between different clusters (which means that points in different clusters are dissimilar from each other) and there are many edges within one cluster (which means that points within the same cluster are similar to each other). Spectral clustering can be derived as an approximation to such graph partitioning problems.
 
But before we need to make simple notations. Two ways of measuring size of clusters are considered:
					\begin{itemize}
						\item $|\mathcal{C}_{i}| = \{\text{number of vertices in $\mathcal{C}_{i}$}\}$
						\item $vol(\mathcal{C}_{i}) = \sum \limits_{i \in \mathcal{C}_{i}}d_{i}$
					\end{itemize} 

where $d_i$ denotes the degree of vertex $i$. Define also:

\[
W(\mathcal{C}_p,\mathcal{C}_q):= \sum \limits_{i \in \mathcal{C}_p,j \in C_q}  a_{ij}\]

Then the simplest and most direct way to construct a partition of the graph is to solve the MinCut problem. In other words, one solves the following optimization problem:

\[
cut(\mathcal{C}_1,\dots,\mathcal{C}_k) = \frac{1}{2}\sum\limits_{i=1}^kW(\mathcal{C}_i,\overline{\mathcal{C}_i}) \rightarrow \min \limits_{\mathcal{C}_1,\dots,\mathcal{C}_k}
\]

In case of $k=2$ the problem is easy to solve. However, in practice MinCut often does not lead to satisfactory partitions. The problem is that in many cases, the solution of MinCut simply separates one individual vertex from the rest of the graph. Of course this is not what is wanted to achieve in clustering, as clusters should be reasonably large groups of points. The two most common objective functions to encode this are the following:

\[
RatioCut(\mathcal{C}_1,\dots,\mathcal{C}_k) = \sum \limits_{i=1}^k \frac{cut(\mathcal{C}_i,\overline{\mathcal{C}_i})}{|\mathcal{C}_{i}|} \rightarrow \min \limits_{\mathcal{C}_1,\dots,\mathcal{C}_k}
\]

\[
Ncut(\mathcal{C}_1,\dots,\mathcal{C}_k)= \sum \limits_{i=1}^k \frac{cut(\mathcal{C}_i,\overline{\mathcal{C}_i})}{vol(\mathcal{C}_{i})} \rightarrow \min \limits_{\mathcal{C}_1,\dots,\mathcal{C}_k}
\]

Unfortunately, introducing balancing conditions makes the previously simple to solve MinCut problem become NP hard. Hence, several relaxation techniques are proposed to solve such problems. Spectral clustering is one of the ways. Relaxing Ncut leads to normalized spectral clustering, while relaxing RatioCut leads to unnormalized spectral clustering. In \cite{spectral} it is described how to write decribed above problems as trace minimization problems. As an example, consider $RatioCut$ minimization problem. Given a partition of vertices $V=\{v_1,\dots,v_n\}$ into $k$ sets $\mathcal{C}_1,\dots,\mathcal{C}_k$, authors introduce $k$ indicator vectors $h_j = (h_{1,j},\dots,h_{n,j})^T$ as:
\[
h_{i,j} = \begin{cases}
\frac{1}{\sqrt{|\mathcal{C}_j|}} & \text{if $v_i \in \mathcal{C}_j$}\\
0 & \text{otherwise}
\end{cases}
\]

The matrix $H$ is the matrix that contains those $k$ indicator vectors as columns. Then the equivalent problem is written as:

\[
\min \limits_{\mathcal{C}_1,\dots,\mathcal{C}_k} Tr(H^TLH) \
s.t. \ H^TH=I, \ H \text{ - defined above}
\]

Now the problem is relaxed by allowing the entries of the matrix $H$ to take arbitrary real values. Then the relaxed problem becomes easy:

\[
\min \limits_{H \in \mathbb{R}^{n \times k}} Tr(H^TLH) \
s.t. \ H^TH=I
\]

Some theory behind tells that the solution is given by choosing $H$ as the matrix which contains the first $k$ eigenvectors of $L$ as columns. To re-convert the real valued solution matrix to a discrete partition standard way is to apply the $k$-means algorithms on the rows of $H$. Similar technique is provided for approximating $Ncut$ (but here one obtains normalized Laplacian). Another way to explain spectral clustering is based on random walks on the graph. A random walk on a graph is a stochastic process which randomly jumps from vertex to vertex. In the paper it is shown that spectral clustering can be interpreted as trying to find a partition of the graph such that the random walk stays long within the same cluster and seldom jumps between clusters. Finally, one obtains three possible algorithms to provide spectral clustering:

\begin{algorithm}[H]
    \caption{Unnormalized Spectral Clustering}
    \label{ncg_alg}
	\SetAlgoLined

	\KwIn{adjacency matrix $A$, number of clusters $k$} %% ����� ����� ������� �������� ���������
	\KwOut{clusters $\mathcal{C}_1,\dots,\mathcal{C}_k$} %% ��������� ������ ���������
	Given adjacency matrix $A$, compute its Laplacian $L$\\
	
	Compute the first $k$ eigenvectors $u_1, \dots , u_k$ of $L$. Let $U \in \mathbb{R}^{n\times k}$ be the matrix containing the vectors $u_1 , \dots, u_k$ as columns\\
		
    \For{$i = 1,\dots,n$}
    {
        Let $y_i \in \mathbb{R}^k$ be the vector or the $i$-th row of $U$
    }

	Cluster the points $(y_i),i=1,\dots,n$ with the $k$-means algorithm into clusters $\mathcal{C}_1,\dots,\mathcal{C}_k$

    \Return $\mathcal{C}_1,\dots,\mathcal{C}_k$	
\end{algorithm}


\begin{algorithm}[H]
    \caption{Normalized Spectral Clustering with $L_{rw}$}
    \label{ncg_alg}
	\SetAlgoLined

	\KwIn{adjacency matrix $A$, number of clusters $k$} %% ����� ����� ������� �������� ���������
	\KwOut{clusters $\mathcal{C}_1,\dots,\mathcal{C}_k$} %% ��������� ������ ���������
	Given adjacency matrix $A$, compute its normalized Laplacian $L_{rw}$\\
	
	Compute the first $k$ eigenvectors $u_1, \dots , u_k$ of $L_{rw}$. Let $U \in \mathbb{R}^{n\times k}$ be the matrix containing the vectors $u_1 , \dots, u_k$ as columns\\
		
    \For{$i = 1,\dots,n$}
    {
        Let $y_i \in \mathbb{R}^k$ be the vector or the $i$-th row of $U$
    }

	Cluster the points $(y_i),i=1,\dots,n$ with the $k$-means algorithm into clusters $\mathcal{C}_1,\dots,\mathcal{C}_k$

    \Return $\mathcal{C}_1,\dots,\mathcal{C}_k$	
\end{algorithm}

\begin{algorithm}[H]
    \caption{Normalized Spectral Clustering with $L_{sym}$}
    \label{ncg_alg}
	\SetAlgoLined

	\KwIn{adjacency matrix $A$, number of clusters $k$} %% ����� ����� ������� �������� ���������
	\KwOut{clusters $\mathcal{C}_1,\dots,\mathcal{C}_k$} %% ��������� ������ ���������
	Given adjacency matrix $A$, compute its normalized Laplacian $L_{sym}$\\
	
	Compute the first $k$ eigenvectors $u_1, \dots , u_k$ of $L$. Let $U \in \mathbb{R}^{n\times k}$ be the matrix containing the vectors $u_1 , \dots, u_k$ as columns\\

Form the matrix $T \in \mathbb{R}^{n\times k}$ from $U$ by normalizing the rows to $l_1$-norm\\		
		
    \For{$i = 1,\dots,n$}
    {
        Let $t_i \in \mathbb{R}^k$ be the vector or the $i$-th row of $T$
    }

	Cluster the points $(t_i),i=1,\dots,n$ with the $k$-means algorithm into clusters $\mathcal{C}_1,\dots,\mathcal{C}_k$

    \Return $\mathcal{C}_1,\dots,\mathcal{C}_k$	
\end{algorithm}

Several comments about this derivation of spectral clustering should be made. Firstly, there is no guarantee whatsoever on the quality of the solution of the relaxed problem compared to the exact solution. Secondly, the relaxations discussed are not unique.  The reason why the spectral relaxation is so appealing is not that it leads to particularly good solutions. Its popularity is mainly due to the fact that it results in a standard linear algebra problem which is simple to solve.
		
\section{Data}

	In our experiments we consider standard real-world graphs that are often used for testing community detection algorithms, such as ''Zachary's karate club'', ''American college football'', ''Books about US politics''. For this graphs we know true clusters. The description of these graphs can be found in internet.

	Also we have generated an artificial graph using stochastic block model.

\section{Experimental results}
	We launch all described algorithms on our graphs and compare their performance using different metrics, namely: NMI, Recall, Precision, F1-score, Normalized cut, Modularity.
	The higher this metrics, the better (except for normalized cut). We can also mention that NMI, Recall, Precision, F1-score are based on comparison of predicted clustering with ground-truth clustering (the value $1.0$ corresponds to perfect matching), while Normalized cut and Modularity mesures the quality of obtained partition and don't require true labels.

	The results are presented in ''Experiment.ipynb''. It's difficult to determine the best algorithm, because the results depend on graph and quality metric very much.


\section{Work split}

Marina Danilova has studied the article about SDP and has implemented corresponding functions in ''Lib/SDP.py''. 
Also she prepared the section in this report and slides for presentation about this method.

Alexander Podkopaev has studied the article about spectral clustering and has implemented corresponding functions in ''Lib/MinCut.py''.
Also he prepared the section in this report and slides for presentation about this method as well as the part with auxilary code for loading and processing graphs in ''Lib/Load.py'' and ''Lib/Transformations.py''

Nikita Puchkin has studied the article about natural conjugate gradient method and has implemented corresponding functions in ''Lib/NCG.py''.
Also he prepared the section in this report and slides for presentation about this method. ''Experiment.ipynb'' and slides with experimental part were prepared by him as well.

Igor Silin has studied the article about greedy modularity-based method and has implemented corresponding functions in ''Lib/modularity.py''.
Also he prepared the section in this report and slides for presentation about this method. Part ''Introduction to community detection'' in this report and in presentation  was prepared by him as well, as well as auxilary code for computing quality metrics and other technical functions programmed ''Lib/Metrics.py''.




\renewcommand{\refname}{References}
\begin{thebibliography}{9}
	\bibitem{sdp}
	On semidefinite relaxations for the block model, Arash A. Amini and Elizaveta Levina, 2016.
    \bibitem{ncg}
    A fast inference algorithm for stochastic blockmodel, Zhiqiang Xu, Yiping Ke, Yi Wang, 2014.
    \bibitem{modularity}
    Fast algorithm for detecting community structure in networks, M.E.J.Newman, 2003.
    \bibitem{spectral}
    A Tutorial on Spectral Clustering, Ulrike von Luxburg, 2007.
\end{thebibliography}

\end{document}
